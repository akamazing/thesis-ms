\section{Testing Machine Learning Programs}
In absence of a reliable oracle to indicate the correct output for arbitrary inputs, machine learning programs are often very hard to test. The general term for such softwares that do not have a reliable test oracle is “non-testable programs”. Such programs can be tested in one of the two ways:
Creating multiple implementations of the same program and testing them on same inputs and comparing the results. If the outputs are not same then either of the implementations can contain error. This approach is called pseudo-oracle.
In absence of multiple oracle, metamorphic testing can be used. In metamorphic testing, the input is modified using a metamorphic relation such that the two sets of input will generate similar outputs. If similar outputs are not observed then there must be a defect.
The main challenge with metamorphic testing is to come up with the metamorphic relations to transform inputs since such coming up with such relations require domain knowledge and/or familiarity with the implementation.
In this paper the authors seek to create a taxonomy of metamorphic relationships that can be applied to the input data for both supervised and unsupervised machine learning softwares. These set of properties can be applied to define the metamorphic relationships so that metamorphic testing can be used as a general testing method for machine learning applications. The problem with some of the current machine learning frameworks like: Weka and Orange is that they compare the quality of results but don’t evaluate the correctness of the results. The authors apply metamorphic testing to three ML applications: MartiRank, SVM-Light, PAYL.
MartiRank is a supervised ML algorithm that applies segmentation and sorting of the input data to create a model. The algorithm then performs similar operations from the model on the test data to produce a ranking list. SVM-Light is an open-source implementation of SVM that also has a ranking mode. The authors also investigated an intrusion detection system called PAYL. PAYL is an unsupervised machine learning system. It’s dataset simply consist of TCP/IP network payloads(stream of bytes) without any label or classification.
Based on the analysis of MartiRank algorithm, the authors realized that the actual values of the attributes were not very important but their relative values determined the model. Thus, adding a constant value to every attribute or multiplying each attribute with a positive number, should not affect the model and generate the same ranking as before. Thus, the metamorphic properties identified were: addition and multiplication. Applying the model on two sets of data, one of which created from the other, either by multiplying a positive number or, adding a constant number, should not change the ranking. Changing the order of examples should not affect the model or ranking since the algorithm sorts the inputs thus, MartiRank also has permutative metamorphic property. Multiplying the data by a negative constant value will create a new sorting order which can easily predicted. The only change to the model will be the sorting direction i.e. the algorithm will change the sorting direction but keep the sorting order intact. Thus, MartiRank also displays an invertive metamorphic property where the output can be predicted by taking the opposite of input. MartiRank also includes inclusive and exclusive metamorphic properties. Knowing the model can help predict the position of any new elements.
